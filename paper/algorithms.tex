% !TEX root = ./cvl.tex
\section{Algorithms}
\label{sec:algorithms}


Each record $r \in O$ correponds to a record $r' \in I$, and for each record $r' \in I$ there may be up to $\mathcal{Z}$ records in $O$ representing it at different scales. To make things easier to explain we consider these records to be logically the same record.

For each record $r \in I$, the record is either represented at zoom-level $z$ or it has been deleted at a higher zoom-level $z': z < z' < \mathcal{Z}$.

A property of the algorithm is that the records stored for a zoom-level $z$ are a subset of the records stored at zoom-level $z+1$. This means that the \emph{zoom-consistency} constraint~\cite{fusiontables} is automatically enforced by the algorithm.

The CVL algorithm computes the zoom-levels in reverse order, i.e. by $z = \mathcal{Z}-1, \dots, 0$. At each zoom-level there are two main task performed by the CVL algorithm: \emph{finding conflicts} and \emph{resolving conflicts}. Before we can talk about conflicts, we need to define cartographic constraints in the context of CVL, as the notion of a conflict depends upon this definition.

\subsection{Cartographic constraints in CVL}

In CVL a cartographic constraint is a condition that must hold for all record subsets of a given size. Constraints are evaluated separately for different partitions and repeatedly at each zoom-level.

We formally consider the following two constraints in our work, but note that with our implementation of CVL it is easy to define new constraints. 

\begin{description}
\item [Proximity] Any two records must separated by at least $d$ pixels
\item [Visibility] Given a uniform grid of cells at most $K$ records may intersect any given cell
\end{description}

\subsection{Conflicts and hitting sets}

Given this brief introduction to constraints, we can now introduce the notion of a conflict. 
A conflict is a record subset for which a given constraint does not hold. As constraints are formulated for record subsets of a given size, it follows that a conflict can be resolved by deleting some of the records in the conflict from the current zoom-level.


\subsection{Cartographic constraints and conflicts}

We use two related terms to deal with selection of records at multiple zoom-levels: Constraints and conflicts. A \emph{constraint} states a condition that must hold over a set of records. A \emph{conflict} represents a subset of records that together violate a constraint. A conflict can always be resolved by deleting some of the records, indicated by the \emph{degree} of the conflict.

\subsubsection{Example}
Consider a proximity constraint that states that every pair of records must be separated by some minimum distance. Evaluating this constraint over a set of records produces a conflict for each a pair of records that are too close together. For each conflict, one of the records must be deleted in order to respect the constraint.

\subsection{Constraints}

