%\documentclass{vldb}
\documentclass[10pt,conference,letterpaper]{IEEEtran}

\usepackage{times}
\usepackage{graphicx}
\usepackage{listings}
\usepackage{color}
\usepackage{url}
\usepackage{amsmath}


\sloppy

\newcommand{\marcos}[1]{\ \\ \fbox{\parbox{1.0\linewidth}{{\sc Marcos}:\\ #1}}}

\newcommand{\minisec}[1]{\noindent\textbf{#1.}}


%%%%%%%%%%%%
% MVS: Language definitions
%
\renewcommand{\ttdefault}{pcr}
\lstset{
  basicstyle=\small\ttfamily,
  breaklines=true
}
\lstdefinelanguage{cvl}{
  morekeywords={generalize,to, with, id, geometry, other, at, zoom, levels, rank, by, subject, and, create, constraint, as, not, exists, resolve, if, delete, select, from, where, in, order, over,partition, merge, partitions, setup, teardown,force,min,level,for,allornothing},
  sensitive=false,
  morecomment=[l]{//},
  morecomment=[s]{/*}{*/},
  morestring=[b]",
}
\lstset{
  language=cvl
}
%%%%%%%%%%%%

\title{Declarative Cartography: In-Database Map Generalization of Geospatial Datasets}

\author{
{Pimin Konstantin Kefaloukos{\small $~^{\#,*,+}$}, Marcos Vaz Salles{\small $~^{\#}$}, Martin Zachariasen{\small $~^{\#}$} }%
\vspace{1.6mm}\\
\fontsize{10}{10}\selectfont\itshape
$^{\#}$\,University of Copenhagen \hspace{3ex} $^{*}$\ Grontmij A/S \hspace{6ex} $^{+}$\ Geodata Agency (GST)         \\
Copenhagen, Denmark \hspace{5ex} Glostrup, Denmark \hspace{6ex} Copenhagen, Denmark\\
\fontsize{9}{9}\selectfont\ttfamily\upshape
\{kostas, vmarcos, martinz\}@diku.dk
}

\begin{document}
\maketitle

\begin{abstract}
Creating good maps is the challenge of cartographic generalization. An important generalization method is selecting subsets of the data to be shown at different zoom-levels subject to a set of cartographic constraints. For example, in a tourist attraction rating system, one needs to distinctively visualize important attractions, and constrain object proximity to allow space for user interaction. On the other hand, in a journalistic piece that maps traffic incidents, maintaining the underlying distribution of data is the most important aspect, but at the same time object density must be constrained to ensure high-performance data transfer and rendering.

Unfortunately, with current tools, users must explicitly specify which objects to show at each zoom level of their map, while keeping their application constraints implicit. This paper introduces a novel declarative approach to cartographic generalization based on a language called CVL, the Cartographic Visualization Language. In contrast to current tools, users declare application constraints and object importance in CVL, while leaving the selection of objects implicit. In order to compute an explicit selection of objects, CVL scripts are translated into an algorithmic search task. We show how this translation allows for reuse of existing algorithms from the optimization literature, while at the same time supporting fully pluggable, user-defined constraints and object weight functions. In addition, we show how to evaluate CVL entirely inside a relational database. The latter allows users to seamlessly integrate storage of geospatial data with its transformation into map visualizations. In a set of experiments with a variety of real-world data sets, we find that CVL produces generalizations in reasonable time for off-line processing and with quality close to optimal. 
\end{abstract}

% !TEX root = ./cvl.tex
\section{Introduction}
In recent year, two major cross-cutting trends in data management have been  big data and crowd-sourcing. These general trends have been changing a lot of fields, and the field of automated cartographic generalization is no exception. This field is concerned with the production of multi-scale maps from geographical datasets, where the main concern is the selection and presentation of data at different zoom-levels.

In this work we present CVL (pronounced \emph{civil}), a language and implementation for cartographic generalization, which is a reaction to these trends. The heart of CVL is a generalization procedure that runs entirely in the database.

As a product of the big data phenomenon, data volumes are now growing significantly faster than our ability to move data around, leading to \emph{diminishing relative bandwidth}. This condition has lead to a practice of moving code to data~\cite{mapreduce} instead of vice versa~\cite{fusiontables}. Our approach to automated cartographic generalization is a reaction to diminishing relative bandwidth. CVL is designed to compile to other languages that can run where the data is stored, e.g. SQL or MapReduce.

Humans are increasingly collaborating and sharing online, a phenomenon known as crowd-sourcing. In recent years this has produced monumental value in many areas. An effect of crowd-sourcing is that a multitude of general purpose background maps of the world can now be downloaded for free~\cite{openstreetmap,googlemaps,bingmaps}. As a consequence of this, we consider the application of automated cartographic generalization to general background maps to be no longer relevant, at least for a very large class of important use cases of maps.

This does not mean that automated cartographic generalization is a closed chapter in general, but that the relevance has shifted away from general purpose background maps to other kinds of maps such as \emph{overlays}. As a consequence of the big data and crowd-sourcing trends, such datasets are now flourishing! This is the kind of data targeted by CVL. While the general task of generalization has proven hard, there are indications that generalizing overlays is easier, and recent results in this field show promise~\cite{fusiontables,thatotherpaper}. 

While fully automatic generalization methods have been demonstrated to work on these types of data, they work best for simple point datasets. With CVL we take a small step back towards giving humans control of the generalization process. Just enough to improve the quality over fully automatic methods, but not so much as to require humans to waste a lot of time managing every detail of the generalization process~\cite{fme}. We believe that human-computer interaction should ideally be a dialogue between the human and the computer, rather than a monologue by one part.

We make the following contributions:

\begin{itemize}
\item A declarative language for generalizing spatial datasets called CVL, which is designed to be simple and efficient to use for non-cartographers. 
\item A mapping of the the cartographic generalization problem to the multi-set cover problem, which makes constraints fully pluggable and allows reuse of well-known algorithms.
\item A compilation procedure of CVL to a relational database backend. This enables us to reuse basic database technology for data management and scalability.
\end{itemize}

In Section~\ref{sec:background} we cover some background for automated cartographic generalization. In Section~\ref{sec:cvl-language} we introduce the CVL language. In Section~\ref{sec:algorithms} we introduce the mapping to multi-set cover problem and revisit algorithms for this problem. In Section~\ref{sec:compilation} we introduce the compilation procedure, which enables us to run CVL on a relational database backend.

% !TEX root = ./cvl.tex
\section{Multi-scale filtering Problem}
\label{sec:background}

\martin{I am updating this section. Introduce terms (e.g. geospacial records, weights, zoom levels, cells, conflicts, constraints, etc.}

%\marcos{We should clearly state what the problem is in this section, in a separate subsection.}

\marcos{Here is where your basic definitions go, e.g., what is a cell, what are zoom levels, how are objects typically selected when creating a map, what are basic and implicit constraints, such as adjacency and zoom consistency, what are application-specific constraints, what is importance/weight, etc.}

\subsection{Conflicts and conflict sets}
\label{sec:conflicts}
\kostas{move to background}
The principle of constant information density implies that we must choose a subset of the records to be visible at each zoom-level of a map in order to satisfy a constraint. Of couser we only have to choose between records whenever a cartographic constraint is actually violated. In our work we use the term \emph{conflict set} to mean a set of records that together are in \emph{conflict}. Records are in conflict when they cause a specific cartographic constraint to be violated at a given zoom-level. Let us consider the proximity constraint introduced before. Using this constraint, there is a conflict for all pairs of records that are less than $d$ pixels apart, and each conflict set consists of one of these pairs. A record can be in several conflict sets if it is too close to more than one record.
%\marcos{This section should explain the core idea of CVL (implicit vs. explicit), and the core idea of each statement in CVL (generalize, create constraint). You may also need a forward pointer to the semantics of the language in the Optimization Models section.}

%\kostas{Move the following to related work once Marcos is done}
%Rule-based languages like Styled Layer Descriptor (SLD)  and Mapnik XML serve a similar purpose as CVL, but using a different approach. The user explicitly decides the filtering of records at each zoom level and how records are presented. CVL is only concerned with the filtering, but is implicit about the exact zoom-level at which a record will appear.

% \marcos{\emph{Condition} in the paragraph above not so clear? }


\subsubsection{Cartographic constraints in CVL}
\label{sec:cartographic-constraints-in-cvl}
\kostas{move to background}
A cartographic constraint in CVL is a condition that must hold for all subsets of a given size. Subsets of records for which the condition does not hold, are said to be in conflict. As part of formulating a constraint, the user writes SQL that finds conflicts for this constraint. Part of the CVL formulation of the proximity constraint is an SQL statement that finds all records that are too near each other at a given zoom-level. The contract between the user and the CVL framework is that the user code must generate $\langle cid, rid \rangle$ tuples that represent the conflict sets. The sematics are that $rid$ is the ID of a record which is a member of a conflict set uniquely identified by $cid$. As an example, the user code for the proximity constraint generates two tuples for each conflict found.

Given the set of conflict tuples generated by the user constraint code, the framework must decide how to resolve the conflicts. A conflict in CVL can always be resolved by removing a subset of the constituent records from the given zoom-level. How many records needs to be removed is also specified in the user code together with the code that finds the conflicts. For the proximity constraint we have to delete one record to resolve each conflict. How this is defined as user code is explained in Section~\ref{sec:create-constraint-statement}.

An example of a cartographic constraint in CVL is the \emph{proximity constraint} which states that at all zoom-levels all visible records must be separated by at least $d$ pixels. CVL does not generally need the user to specify at which zoom-level a record is too close to other records, only what the user considers to be "too close". The formulation of the proximity constraint implies that at each zoom-level a (possibly empty) subset of records must be removed in order to respect the constraint, and this is a general property of constraints in CVL. Which records are prioritized over others is controlled by assigning weights to records to indicate their importance. The implicit goal of evaluating a CVL query is to maintaining as much aggregate weight at each zoom-level as possible while satisfying all constraints.

%\marcos{What about the notion of conflicts and conflict sets? Those are pretty important.}

\subsubsection{Weighting records}
\kostas{move to background}
Records are assigned a weight by user defined code which guides how the CVL framework will resolve conflicts.  The intuitive notion of weight is that it represents the relative importance of a record. In general, the CVL framework will try to find a solution for each zoom-level such that all constraints are  satisfied and such that the aggregate weight of records that are removed is minimized. For example, if two records constitute a conflict with respect to the proximity constraint, the CVL framework will delete the one with lesser weight, unless deleting the record with higher weight yields are better global solution for that zoom-level. In other words, the user does not control directly how the framework resolves conflicts, but influences the decision by assigning record weights. 

While the user does not need to individually assign weights to records, CVL offers a very flexible scheme for doing this. Weights are assigned by evaluating a SQL expression for each input row. For example a given column in the input database can be used directly as the weight of a record, or the length or area of the record geometry could be used. In fact any floating point expression can be used to weigh records, and it is perfectly ok to use the output of the random number generator or even a constant. This implies that record weight is a partial order, i.e. any number of records can have the same weight.

The only contract between the user and the CVL framework is that the user code must assign a floating point number to each record to use a the weight.

%\marcos{While I can certainly see that WMS@GST uses the process above, it is not clear that the three steps above are an accurate representation of all map services?}
%\marcos{It seems to me that it is important to say that rendering is orthogonal to the work, and that the selection task of generalization is computationally intensive and can be done at indexing time. Perhaps these two points can be conveyed with less words than in the section above and the section below? That would leave us space to introduce other important basic definitions listed in the comment at the beginning of this section.}

% text is moved to introduction /MZ


% !TEX root = ./cvl.tex
\section{CVL Language}
\label{sec:cvl:language}
The Cartographic Visualization Language (CVL) is a declarative language that can be used to specify an instance of the multi-scale filtering problem. CVL is a rule-based language with a similar goal as other rule-based languages for filtering spatial datasets~\cite{sld,mapnik}, i.e. to control the density of information at each zoom-level. The CVL approach is however markedly different. In the related languages the user must explicitly control the filtering of records at each zoom-level, while also specifying how records are rendered (presentation). First of all, CVL is not concerned with presentation, only filtering. Furthermore CVL controls filtering in a novel constraint-based way. Instead of having the user explicitly control the filtering of records at each zoom-level, CVL lets the user choose \emph{cartographic constraints} that are instead enforced at all zoom-levels. By making the constraints explicit and the control implicit a very concise formulation is obtained. See Figure~\ref{fig:cvl:example:airports} for an example of this.

CVL is one of the first languages and frameworks to implement the vision of reverse data management~\cite{meliou2011reverse}. In reverse data management, the core idea is that a user states a set of constraints and objective. These are given together with an input database to an optimization algorithm which computes an output database that is feasible and optimal with regard to the constraints and objective (if a feasible solution exists). This is exactly how CVL works. Furthermore a feasible solution is guaranteed to exist as filtering out all records is always a feasible solution.

The CVL language has two statements, the \emph{generalize} statement (see Section~\ref{sec:generalize:statement}) and the \emph{create-constraint} statement (see Section~\ref{sec:create:constraint:statement}). The create constraint statement is used to formulate new cartographic constraints and the generalize statement is used for everything else. 

The CVL language builds on top of SQL and reuses SQL as a language for formulating constraints and record weighting schemes.

\subsection{Generalize statement}
\label{sec:generalize:statement}

The generalize statement is the main statement in CVL. This statement creates a new multi-scale dataset from an input table of geospatial records, subject to user defined constraints. The syntax is shown in Figure~\ref{fig:generalize:syntax}. The statement has several clauses, beginning with the specification of input and output tables. Instead of giving the name of an input table, the user can optionally write a dynamic select statement in SQL on the form \texttt{(SELECT ...) t}. The next clause is the \emph{zoom-levels} clause where the user writes a positive integer, which is the zoom-level  at which the filtering process will begin. The \emph{weigh-by} clause is used to give an arbitrary floating point expression that is evaluated for each row in the input and used as weight for that record. The \emph{subject-to} clause lists the cartographic constraints along with any parameters (as a comma-separated list). The \texttt{AND} keyword is used to separate constraints in the case more than one is used.

For clarity the syntax is shown without two clauses, the \emph{with-id} clause and the \emph{with-geometry} clause, which are used simply to specify the names of the ID and geometry columns in the input table.

\begin{figure}[htbp]
\begin{center}
\begin{lstlisting}
GENERALIZE 
   {input} TO {output}
AT {integer} ZOOM LEVELS
WEIGH BY
  {float expression}
SUBJECT TO 
   {constraint} {float parameters} [AND
   {constraint} {float parameters} [AND
   ...]]
\end{lstlisting}
\caption{Syntax of generalize statement}
\label{fig:generalize:syntax}
\end{center}
\end{figure}

\begin{figure*}[tb]
  \begin{minipage}{0.329\linewidth}
    \centerline{\includegraphics[width=0.95\linewidth]{./figs/airports.png}}
    \centerline{(a) Full Openflights Airport dataset}
  \end{minipage} \hfill
  \begin{minipage}{0.329\linewidth}
    \centerline{\includegraphics[width=0.95\linewidth]{./figs/airports_z4.png}}
    \centerline{(b) Airports on zoom-level 5}
  \end{minipage} \hfill
  \begin{minipage}{0.329\linewidth}
    \centerline{\includegraphics[width=0.95\linewidth]{./figs/airports_z0.png}}
    \centerline{(c) Airports on the top-most zoom-level.}
  \end{minipage}
  \vspace{-0ex}
  \caption{Airport map (7K points) before (a) and after (b, c) running CVL. The output corresponds to the CVL statement in Figure~\ref{fig:cvl:example:airports}.}
  \label{fig:graphical:output:airport}
  \vspace{-2ex}
\end{figure*}

An example of generalizing a dataset using the generalize statement is shown in Figure~\ref{fig:cvl:example:airports}. In this example a dataset containing point records representing the location of airports world-wide is generalized. The records are weighted by using the name of a column containing the number of routes departing from each airport (CVL automatically handles the cast from integer to floating point). The intuition is that airports with more departures are more important. The single constraint that is enforced is the density constraint, with a parameter of $K=16$. Recall that the density constraint says that each tile can contain at most $K$ records.

\begin{figure}[htbp]
\begin{center}
\begin{lstlisting}
GENERALIZE 
   airports TO airports2
AT 18 ZOOM LEVELS
WEIGH BY
  num_departures
SUBJECT TO 
   density 16 
\end{lstlisting}
\caption{Generalization of airports. The airports are weighting by number of departures. See Figure~\ref{fig:graphical:output:airport} for a vizualization of the result.}
\label{fig:cvl:example:airports}
\end{center}
\end{figure}

The resulting map is shown in Figure~\ref{fig:graphical:output:airport} and has at most $16$ airports on each tile. This implies that some airports are missing on some zoom-levels. The CVL framework automatically gives priority to the airports with the highest weight. How this is done is elaborated upon in sections~\ref{sec:optimizationmodel} and~\ref{sec:algorithms}.

\subsection{Create constraint statement}
\label{sec:create:constraint:statement}

% change text
Cartographic constraints are defined using the create-constraint statement.  The basic syntax of the statement is shown in Figure~\ref{fig:create:constraint:syntax}. The body of the statement is an SQL select statement that selects tuples $\langle cid, rid\rangle$ representing conflicts that are found at a given zoom-level in the map. See Section~\ref{sec:conflicts} for the semantics of the tuples.

The \emph{resolve-if-delete} clause is used to compute the integer number of records that must be deleted in order to resolve the conflict with a given $cid$. For the proximity constraints this number is always 1, but for other constraints this may vary.

\begin{figure}[htbp]
\begin{center}
\begin{lstlisting}
CREATE CONSTRAINT C1
AS NOT EXISTS
  {SQL select statement}
  
RESOLVE cid IF DELETE (
  {integer expression}
)
\end{lstlisting}
\caption{Syntax of create constraint statement}
\label{fig:create:constraint:syntax}
\end{center}
\end{figure}

Using this syntax, the definition of the proximity constraint is given in Figure~\ref{fig:proximity:definition}. The body of the constraint is a distance self join using a distance function \texttt{ST\_Distance} provided by a spatial extension to SQL. This join finds all pairs of records that are too close, i.e. less than $10$ pixels apart if running the example from Figure~\ref{fig:cvl:example:airports}. For each conflict the select statement outputs two tuples and exactly once for each conflict. The resolve-if-delete clause is simply the constant $1$, because that is how many records must be deleted to resolve a proximity conflict.

\begin{figure}[htbp]
\begin{center}
\begin{lstlisting}
CREATE CONSTRAINT Proximity
AS NOT EXISTS (
  SELECT 
    l.{rid} || r.{rid} AS cid,
    Unnest(array[l.{rid}, r.{rid}]) AS rid
  FROM
    {level_view} l
  JOIN
    {level_view} r
  ON
    l.{rid} < r.{rid}
  AND
    l.{geom} && ST_Expand(r.{geom}, 
      CVL_Resolution({z}, 256) * 
        {parameter_1})
  AND
    ST_Distance(l.{geom}, r.{geom}) <
      CVL_Resolution({z}, 256) * {parameter_1}
)

RESOLVE cid IF DELETE (
  1
)
\end{lstlisting}
\caption{Definition of the proximity constraint}
\label{fig:proximity:definition}
\end{center}
\end{figure}

In Figure~\ref{fig:proximity:definition} you will notice some names enclosed in curly braces like \texttt{\{rid\}}. These are variables which are bound at runtime by the CVL framework and are intended for making the definition of constraints simpler. The variables \texttt{\{rid\}} and \texttt{\{geom\}} are bound to the column names containing the ID and geometry of the records. The \texttt{\{level\_view\}} is bound to a view that contains all records that are visible at the current level, i.e. the records that have not been filtered out at at higher zoom-level. You will also notice a call to a function \texttt{CVL\_Resolution(\{z\}, 256)}. This is one of the utility functions defined by the CVL runtime, also with the purpose of making the definition of constraints simpler. This function returns the resolution (meter/pixel) at zoom-level \texttt{\{z\}}, where \texttt{\{z\}} is a variable bound to the currently evaluated zoom-level. The variable \texttt{\{parameter\_1\}} is a constraint parameter, and would be bound to $10$ if running the example from Figure~\ref{fig:cvl-example-airports}.

In the CVL example in Figure~\ref{fig:cvl:example:airports} we referenced the as yet undefined density constraint. In Figure~\ref{fig:density:definition} we show how this constraint may be defined using CVL. The CVL definition uses an extension of the basic create-constraint syntax, namely the \emph{setup} and \emph{tear down} clauses which will be covered again in Section~\ref{sec:implementation:extensions}. The purpose of these clauses is to enable arbitrary SQL statements to be run before and after the constraint body is evaluated at each zoom-level. During the setup phase we create an auxiliary table called \texttt{busted\_tiles} which contains tuples $\langle tile\_id, rid \rangle$ identifying tiles that are intersected by more than $K$ records, and the ID of those records. The body of the constraint simply iterates over the auxiliary table, using the \texttt{tile\_id} column as the conflict ID.

The user does not need to know how the conflicts are handled, because all conflicts are automatically resolved by the CVL framework using one of the algorithms introduced in Section~\ref{sec:algorithms}.

%In this example, calls are made to other CVL runtime functions, namely \texttt{CVL\_WebMercatorCells} and \texttt{CVL\_PointHash}. These functions returns a set of points corresponding to the centroids of intersected tiles (given a geometry) and a unique identifier for points, respectively. The unique identifier for points is based on the well-known GeoHash algorithm.

\begin{figure}[htbp]
\begin{center}
\begin{lstlisting}
CREATE CONSTRAINT Density
AS NOT EXISTS (
    SELECT
        busted_tiles.cid,
        busted_tiles.rid
    FROM
        busted_tiles
)

RESOLVE cid IF DELETE (
  SELECT count(*) - {parameter_1}
  FROM   busted_tiles bt
  WHERE  bt.cid = cid
)

WITH SETUP (
    CREATE TEMPORARY TABLE busted_tiles AS (
        SELECT
            t.cid,
            Unnest(array_agg(t.cvl_id)) AS rid
        FROM
        (
        SELECT
            CVL_PointHash(CVL_WebMercatorCells({geometry}, {z})) AS cid,
            {rid}
        FROM
            {level_view}
        ) t
        GROUP BY t.cid
        HAVING count(*) > {parameter_1}
    );
    CREATE INDEX busted_tiles_id_idx ON busted_tiles (cid);
)

WITH TEARDOWN (
  DROP TABLE busted_tiles;
)
\end{lstlisting}
\caption{Definition of density constraint}
\label{fig:density:definition}
\end{center}
\end{figure}

% give the variables by example, and explain together with the example, instead of listing everything up front.

%\marcos{These could be introduced by need along with the examples integrated into the sections above.}


% !TEX root = ./cvl.tex
\section{\hl{Selection optimization problem}}
\label{sec:optimizationmodel}

In this section, we formally define the selection problem as an optimization problem. Let $R$ be the set of records in the dataset. Each record $r \in R$ has an associated weight $w_r > 0$ which models the importance of the record. 

Evaluating a CVL query generates a number of conflicts, i.e., all sets of records that violate a constraint. The conflicts are used to formulate constraints in the selection optimization problem. Let $C$ be the set of conflicts. A conflict $c \in C$ is a set of records $R_c \subseteq R$, where at least $\lambda_c \geq 1$ records must be deleted. This is formulated as a set of inequalities (\ref{eq:general-constraints}) in the optimization problem.

The selection problem can now be modeled as a 0-1 integer program. Let $x_r$ be a 0-1 decision for each record $r \in R$ that is 1 if record $r$ is \emph{deleted}, and 0 otherwise. Then \hl{at a single-scale, the} problem can be stated as follows:

\begin{align}
  \label{eq:objective}
  \min ~\sum_{r \in R} &w_r x_r \\
  \label{eq:general-constraints}
  \sum_{r \in R_c} x_r &\geq \lambda_c, ~~~~ c \in C \\
  x_r & \in \{0, 1\}, ~~ r \in R, ~c \in C
\end{align}

The goal (\ref{eq:objective}) is to minimize the total weight of the records that are deleted. This is the \emph{set multicover problem} --- a generalization of the well-known set cover problem where each element needs to be covered multiple times instead of just once~\cite{rajagopalan1998primal}. 

\hl{In our formulation, conflicts correspond to }\textit{elements}\hl{ in the set multicover problem, while records correspond to }\textit{sets}.\hl{ Each conflict must be ``covered'' $\lambda_c \geq 1$ times by choosing a subset of records that are deleted ($x_r=1$). Because the selection of records is modeled using a 0-1 variable, each record can be chosen at most once.}

\hl{This maps the selection problem at a single scale to an NP-hard optimization problem, namely set multicover. However, it can be shown that the selection optimization problem as presented in Section}~\ref{sec:filtering}\hl{ is also NP-hard for all interesting cases. For an informal proof, consider the vertex cover problem. Given a graph $G=(V,E)$, find a minimum size subset of the vertices $S$ such that every edge in $E$ has an endpoint in $S$. Think of the vertices as records and the edges as conflict sets. This problem would then model the selection problem for the special case where all conflict sets had exactly two records (and all records had identical weights).}

\marcos{Should we not map the other way around in the informal proof above?}

\hl{The vertex cover problem is NP-hard, even for very constrained cases. For example, even if $G$ is a planar graph and every vertex has degree at most 3, the problem remains NP-hard}~\cite{garey1977rectilinear}. \hl{In other words, even if the conflict sets only have two records each, and each record is involved in at most 3 conflict sets, the problem remains NP-hard. It is hard to imagine that any interesting application be more restrictive. Note additionally that the cubic vertex cover problem is APX-hard -- which means that it cannot be approximated arbitrarily close unless $P=NP$}~\cite{alimonti2000some}. %So even this restricted version is ``really'' hard.

In the next section, we discuss algorithmic approaches for solving the selection optimization problem. \hl{We include a further discussion on the objective value} (\ref{eq:objective})\hl{ in the experimental evaluation of our approach.}

% !TEX root = ./cvl.tex
\section{Algorithms for single-scale filtering}
\label{sec:algorithms}

The algorithmic framework for the multi-scale filtering problem is illustrated in Figure~\ref{fig:algorithmic-framework}. After an initialization step, we solve $\mathcal{Z}$ single-scale filtering problems --- followed by a finalizing step.

\begin{figure}[htbp]
\begin{center}
\includegraphics[scale=.6]{figs/cvl_stages.pdf}
\caption{The algorithmic framework: At stage $i$ the single-scale filtering problem is solved for the $i'th$ zoom level.}
\label{fig:algorithmic-framework}
\end{center}
\end{figure}

Below we describe two different heuristic algorithms for solving the single-scale optimization problem. Let $n=|C|$ be the number of constraints (or elements in the set multicover problem), and let $m=|R|$ be the number of records (or sets in the set multicover problem). Recall that $R_c \subseteq R$ is the set of records in constraint $c \in C$. The largest number of records in any constraint is $f = \max_{c \in C} |R_c|$, and is called the \emph{maximum frequency}.

\subsection{Static greedy algorithm (SGA)}
\label{sec:algorithms:sga}

In this algorithm we consider each constraint $c \in C$ in turn, and simply choose the $\lambda_c$ records with minimum weight from constraint $R_c$ --- independently of what has been chosen earlier. If the sets $R_c$ are disjoint, the algorithm is clearly optimal. However, in general no approximation guarantee can be provided. The algorithm runs in $O(n f \log f)$ time, as we just need to sort the records by weight for each constraint; alternatively we can sort all records by weight in $O(m \log m)$ time and pick the minimum weight records from the constraints in linear time in the total number of records in all constraints.

\subsection{LP-based greedy algorithm (LPGA)}
\label{sec:algorithms:lpga}

In this algorithm we first solve a linear programming (LP) relaxation of the set multicover problem. This LP-problem is obtained by relaxing the constraint $x_r \in \{0, 1\}$ to $0 \leq x_r \leq 1$. Then we choose all records $r \in R$ for which the LP-solution variable $x_r$ is at least $1 / f$. Intuitively, we round up to 1 all fractional values that are large enough; the remaining fractional variables are rounded down to 0. 

This algorithm provides a feasible solution to the single-scale problem, and the approximation guarantee is $f$~\cite{seevaziranibookforreference}; thus, if $f$ is small, the algorithm provides a good approximation guarantee. As the LP-problem can be solved in polynomial time, the complete algorithm is polynomial.

%\subsection{Dynamic greedy algorithm (DGA)}

%Described in Vazirani 13.2.1. @Martin: Please write here.

%\marcos{I would only introduce DGA if it is actually evaluated in the experiments.}



% !TEX root = ./cvl.tex
\section{Implementation}
This is were we go nuts with SQL.

% !TEX root = ./cvl.tex
\section{Results}
Show performance, scalability and quality.

% !TEX root = ./cvl.tex
\section{Related work}
\label{sec:related}
\marcos{One should make sure to include the items below.}

Basic reference set:

\begin{itemize}

\item Fusion Tables~\cite{sarma2012fusiontables}. 
% zoom consistency, adjacency, visibility; not general constraints. // different types of objectives, but no declarative interface and integration with standard database technology.

\item Summary and most important references from cartographic generalization survey.

\item Reverse data management vision~\cite{meliou2011reverse}

\end{itemize}

Additional references:

\begin{itemize}

\item In-database processing: briefly repeat argument with Pathfinder and Ferry (As we mentioned earlier,...). Increasingly, relational are incorporating whole programming language interpreters and support for spatial data structures~\cite{Blakeley2008:DotNET}. Then also cite MADLib work~\cite{hellerstein12madlib} as well as Ordonez~\cite{ordonez10udf}. But: direct SQL and not higher-level DSL leveraging SQL + analytics / statistics and not spatial.  

\item Say something about how you actually serve the map after it is built with our framework (e.g., the usual references in linearization)~\cite{hilbert1891ueber}.

\item ...

\marcos{Still performing scan of recent literature, just in case.}

\item if it fits, reverse / symbolic query processing.

\end{itemize}

\martin{We should mention that the dynamic greedy algorithm would have improved the running time compared to the LP-based greedy algorithm, while achieving good quality~\cite{rajagopalan1998primal}.}


% !TEX root = ./cvl.tex
\section{Acknowledgements}
We would like to thank employees and management at Grontmij and the Danish Geodata Agency for great discussions and for partially financing the project. We would also like to thank Amazon for providing an AWS in education grant which we used to implement our experimental setup.

% !TEX root = ./cvl.tex
\section{Conclusion}

In this paper, ...

\marcos{Just summarize paper above; related work now includes future work, preventively.}

% The following two commands are all you need in the
% initial runs of your .tex file to
% produce the bibliography for the citations in your paper.
\bibliographystyle{abbrv}
\bibliography{cvl}  % gvl.bib is the name of the Bibliography in this case
% You must have a proper ".bib" file
%  and remember to run:
% latex bibtex latex latex
% to resolve all references


\end{document}
