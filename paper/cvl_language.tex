% !TEX root = ./cvl.tex
\section{CVL language}
\label{sec:cvl-language}
CVL is a declarative language for generalizing a spatial dataset. It is designed to be very concise, typically no more than ten lines of code. CVL is an acronym for \emph{Cartographic Visualization Language} and the syntax and semantics of this language are defined in this section. The language has only two statements, the \emph{generalize} statement and the \emph{create constraint} statement. 

The generalize statement is used to transform an input spatial dataset to a multi-scale output dataset, which is well-suited for display on a zoomable map such as the web maps that have gained immense popularity over the last ten years. 

The create constraint statement is use to augment CVL with new cartographic constraints.

\subsection{Generalize statement}
The generalize statement has several clauses, some of which are mandatory. The first mandatory clause is used to specify the names of the input and output datasets, and includes the names of fields holding the record ID and record geometry. This clause has the following syntax:

\begin{lstlisting}
GENERALIZE 
   {input} TO {output}
WITH ID
   {field}
WITH GEOMETRY 
   {field}
\end{lstlisting}

Optionally the names of additional fields that should be copied from input to output may be given:

\begin{lstlisting}
WITH OTHER {list of fields}
\end{lstlisting}

The second mandatory clause specifies the number of zoom-levels to generalize for. Zoom-levels are assumed to run from 0 (lowest scale) to some $\mathcal{Z}$ (largest scale). As zoom-levels are zero-indexed, the number that is given in the clause is $\mathcal{Z}-1$:

\begin{lstlisting}
AT {integer} ZOOM LEVELS
\end{lstlisting}

A ranking expression (e.g. a user defined function) is optionally given to guide the generalization process and give priority to "important" records. If the clause is left out, records are ranked by the constant $1$:

\begin{lstlisting}
RANK BY {float expr}
\end{lstlisting}

A partitioning expression (e.g. the name of a record field) is optionally given to semantically split the input dataset into subpartitions. Subpartitions are treated separately by the CVL interpreter. If no clause is given, all records are partitioned by the constant $1$. Optionally, the merge clause can be used to recombine partitions into bigger ones:

\begin{lstlisting}
PARTITION BY
   {scalar expr}
MERGE PARTITIONS    
    {partitons}
    AS {scalar expr}
[AND 
   {partitions} 
   AS {scalar expr}]
[...]
[AND 
   * AS {scalar expr}]
\end{lstlisting}

One of the most central clauses is actually optional. The subject-to clause states the cartographic constraints that must hold for each partition at each zoom-level. Constraint are parameterized by a single float-valued expression, typically a constant:

\begin{lstlisting}
SUBJECT TO 
   {constraint} {float parameters} 
[AND
   {constraint} {float parameters}]
[...]
\end{lstlisting}

The user may optionally choose to manually set the minimum zoom-level (integer) for a partition (scalar). Constraints are not evaluated for these partitions:

\begin{lstlisting}
FORCE MIN LEVEL
   {zoom-level} FOR {partition}
[AND
   {zoom-level} FOR {partition}]
[...]
\end{lstlisting}

The user may optionally name a set of finalizing steps to be performed at the end of the generalization process. CVL has support for two such steps: simplification and a novel "all-or-nothing" transformation. The all-or-nothing transformation ensures that either all records in a partition are visible at a given zoom-level or none of them are.

\begin{lstlisting}
TRANSFORM BY
   {operator} [, {operators}]
\end{lstlisting}

