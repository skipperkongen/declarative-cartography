% !TEX root = ./cvl.tex
\section{Introduction}

\marcos{Discuss and think about title.}
\marcos{Introduce GST somewhere.}

In recent year, two major cross-cutting trends in data management have been  big data and crowd-sourcing. These general trends have been changing a lot of fields, and the field of automated cartographic generalization is no exception. This field is concerned with the production of multi-scale maps from geographical datasets, where the main concern is the selection and presentation of data at different zoom-levels.

In this work we present CVL (pronounced \emph{civil}), a language and implementation for cartographic generalization, which is a reaction to these trends. The heart of CVL is a generalization procedure that runs entirely in the database.

As a product of the big data phenomenon, data volumes are now growing significantly faster than our ability to move data around, leading to \emph{diminishing relative bandwidth}. This condition has lead to a practice of moving code to data~\cite{mapreduce} instead of vice versa~\cite{fusiontables}. CVL is designed to compile to other languages that can run where the data is stored, e.g. SQL or MapReduce.

Humans are increasingly collaborating and sharing online, a phenomenon known as crowd-sourcing. In recent years this has produced monumental value in many areas. An effect of crowd-sourcing is that a multitude of general purpose background maps of the world can now be downloaded for free~\cite{openstreetmap,googlemaps,bingmaps}. As a consequence of this, we consider the application of automated cartographic generalization to general background maps to be no longer relevant, at least for a very large class of important use cases of maps.

This does not mean that automated cartographic generalization is a closed chapter in general, but that the relevance has shifted away from general purpose background maps to other kinds of maps such as \emph{overlays}. As a consequence of the big data and crowd-sourcing trends, such datasets are now flourishing! This is the kind of data targeted by CVL. While the general task of generalization has proven hard, there are indications that generalizing overlays is easier, and recent results in this field show promise~\cite{fusiontables,thatotherpaper}. 

While fully automatic generalization methods have been demonstrated to work on these types of data, they work best for simple point datasets. With CVL we take a small step back towards giving humans control of the generalization process. Just enough to improve the quality over fully automatic methods, but not so much as to require humans to waste a lot of time managing every detail of the generalization process~\cite{fme}. We believe that human-computer interaction should ideally be a dialogue between the human and the computer, rather than a monologue by one part.

We make the following contributions:

\begin{itemize}
\item A declarative language for generalizing spatial datasets called CVL, which is designed to be simple and efficient to use for non-cartographers. 
\item A mapping of the the cartographic generalization problem to the \emph{set multicover problem}, which makes constraints fully pluggable and allows reuse of well-known algorithms.
\item A compilation procedure of CVL to a relational database backend. This enables us to reuse basic database technology for data management and scalability.
\end{itemize}

In Section~\ref{sec:background} we cover some background for automated cartographic generalization. In Section~\ref{sec:cvl-language} we introduce the CVL language. In Section~\ref{sec:algorithms} we introduce the mapping to multi-set cover problem and revisit algorithms for this problem. In Section~\ref{sec:compilation} we introduce the compilation procedure, which enables us to run CVL on a relational database backend.