% !TEX root = ./cvl.tex
\section{Selection optimization problem}
\label{sec:optimizationmodel}

In this section, we formally define the selection problem as an optimization problem. Let $R$ be the set of records in the dataset. Each record $r \in R$ has an associated weight $w_r > 0$ which models the importance of the record. 

There are a number of constraints (formulated as conflict sets) that are modeled using CVL. Let $C$ be the collection of conflict sets. A conflict set $c \in C$ is a set of records $R_c \subseteq R$, where at least $\lambda_c \geq 1$ records must be deleted. 

The selection problem can now be modeled as a 0-1 integer program. Let $x_r$ be a 0-1 decision for each record $r \in \bar{R}$ that is 1 if record $r$ is \emph{deleted}, and 0 otherwise. Then the single-scale problem can be stated as follows:

\begin{align}
  \label{eq:objective}
  \min ~\sum_{r \in R} &w_r x_r \\
  \label{eq:general-constraints}
  \sum_{r \in R_c} x_r &\geq \lambda_c, ~~~~ c \in C \\
  x_r & \in \{0, 1\}, ~~ r \in R, ~c \in C
\end{align}

The goal (\ref{eq:objective}) is to minimize the total weight of the records that are deleted. Constraints~(\ref{eq:general-constraints}) model the conflict sets $c \in C$. This is the \emph{set multicover problem} --- a generalization of the well-known set cover problem where each element/conflict set $c \in C$ needs to be covered $\lambda_c \geq 1$ times by the sets/records $r \in R$; each set/record can be chosen at most once~\cite{rajagopalan1998primal}.

In the next section we discuss algorithmic approaches for solving the single-scale problem. Note that although we solve the multi-scale problem as a series of single-scale problems, in the experimental evaluation of our approach also includes a discussion on the objective value (\ref{eq:objective}) of the \emph{multi-scale} problem. 