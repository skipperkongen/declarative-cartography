\documentclass[11pt, oneside]{article}   	% use "amsart" instead of "article" for AMSLaTeX format
\usepackage{geometry}                		% See geometry.pdf to learn the layout options. There are lots.
\geometry{letterpaper}                   		% ... or a4paper or a5paper or ... 
%\geometry{landscape}                		% Activate for for rotated page geometry
%\usepackage[parfill]{parskip}    		% Activate to begin paragraphs with an empty line rather than an indent
\usepackage{graphicx}				% Use pdf, png, jpg, or eps with pdflatex; use eps in DVI mode
								% TeX will automatically convert eps --> pdf in pdflatex		
\usepackage{amssymb}

\title{Working document for revision of ICDE '14 paper}
\author{Kostas, Martin, Marcos}
%\date{}							% Activate to display a given date or no date

\begin{document}
\maketitle

\section{About}
Notes and overall plan recorded in Kostas/Marcos Skype chat October 16 2013. Citations for ladder/star approach in Kostas/Pia chat November 12 2013. Citations for NP-hardness in Kostas/Martin email.

\section{Formulations for paper}

\subsection{Refinements based on knowledge gained in Z\"{u}rich}

\begin{description}
\item[Typology of constraints (tags: method, related work, discussion)] Various typologies of constraints have been discussed in the literature~\cite{beard1991constraints,harrie2007modelling}. In the typology of Harrie and Weibel~\cite{harrie2007modelling}, we consider only \emph{legibility constraints},  e.g. proximity and density constraints, while not handling \emph{appearance constraints}. Appearance constraints serve the purpose of maintaining representation of the input data. Instead, we use an objective function to balance the legibility constraints, minimizing the combined weight of objects that are removed. 

The spatial distribution of objects is not well maintained by our genealization method. Including appearance constraints~\cite{harrie2007modelling} in the model would be an interesting direction for future work. (Note: Stephan Schmid wrote a nice master's thesis that includes a thorough discussion of constraints~\cite{schmid2008automated}).
\item[Ladder/star approach (tags: method, related work)] When generalizing a dataset for several scales, an important distinction is whether to follow the \emph{ladder} or \emph{star} approach~\cite{foerster2010challenges}. We have only implemented the ladder approach, where lower scales are recursively derived from higher scales. An advantage of using the ladder approach is that the zoom-consistency constraint is automatically satisfied. The star approach is well fitted for use cases where enforcing the zoom-consistency constraint is not appropriate, e.g. when generalizing regional labels were labels appear on intermediate zoom levels; disappearing at lower and/or higher zoom levels.
\end{description}

\subsection{Why map conflict resolution to an NP-hard problem?}

If we model the problem using conflict sets, the problem is
definitely NP-hard for all interesting cases.

Here is why: Consider the vertex cover problem. Given a graph G=(V,E),
find a minimum size subset of the vertices S such that every edge in E
has an endpoint in S. Think of the vertices as records and the edges
as conflict sets. This problem would then model the filtering problem
for the special case where all conflict sets had exactly two records
(and all records had identical weights).

The vertex cover problem is NP-hard, even for very constrained cases.
For example, even if G is a planar graph and every vertex has degree
at most 3, the problem remains NP-hard. In other words, even if the
conflict sets only have two records each, and each record is involved
in at most 3 conflict sets, the problem remains NP-hard. It is hard to
imagine that any interesting application is more restrictive.

Note that the cubic vertex cover problem is APX-hard - which
means that it cannot be approximated arbitrarily close unless P=NP. So
even this restricted version is "really" hard.
%\cite{Some APX-completeness results for cubic graphs, The Rectilinear Steiner Tree Problem is NP-Complete*}




\subsection{Why use conflict sets to model problem?}

We have conflict sets because we want to satisfy constraints. We want to satisfy constraints because it is a common way to reason about feasible solutions to generalization... thinning and "The constraint method for solving spatial conflicts in cartographic generalization"


\section{Question: Why map conflict resolution to an NP-hard problem?}
Here we assume that using conflict sets is a good idea. The question whether that is true is a separate question, that we will also answer.
\begin{itemize}
\item Definition of conflict resolution: Given a set of conflict sets $C$, and a number $\lambda_c < \left\vert c \right\vert$ for each set $c \in C$, choose $\lambda_c$ elements from each set $c$ that will be ``deleted'' 
\item Global optimization version: choose $\lambda_c$ elements of minimum total weight from each set $c$ that will be ``deleted'' 
\item This optimization problem equals set multicover problem, plain and simple. I don't see that we are even ``mapping'' to this problem. It is the same problem?
\item Is there another optimization problem (not minimizing total weight of elements deleted) that involves conflict sets and generates useful solutions to generalization?
\item The SGA algorithm is an exact algorithm for another optimization problem (minimize for each set); but is that really the ``right'' problem to solve and why not?
\end{itemize}

\subsection{Our answer}

\begin{itemize}
\item Actually, for most instances the bulk of the time is spent finding conflicts, not solving the set multicover problem, because in the end we don't solve SMCP
\item We apply approximation algorithms and heuristics which are exact solutions to various other problems that somehow relate to SMCP
\end{itemize}

\section{Question: Why use conflict sets to model problem?}

\begin{itemize}
\item Is there a "conflict free" formulation that would be just as ``good'' and not lead to solving an NP-hard problem?
\item Conflict sets allow a generic way to express many natural user-defined constraints (proximity, visibility)
\item Using conflict sets is a natural way to think about generalization
\item Maybe there is another model that could express proximity and visibility constraints, and which is a natural way to think about generalization, and which leads to an optimization problem that has a polynomial time algorithm?
\end{itemize}

\subsection{Our answer}

\begin{enumerate}
\item We chose conflict sets because it is a natural way to think about generalization and because it allows users to formulate several natural spatial constraints (proximity, visibility etc)
\item We can not think of another model with these properties and which also leads to an optimization problem that has an efficient solution
\item Asking us to prove that such a model does not exist is a tall order...
\end{enumerate}

\section{Questions: Various questions answered in author feedback}

Use this section to capture some compact formulations that can be plugged into paper

\bibliographystyle{abbrv}
\bibliography{revision}  % gvl.bib is the name of the Bibliography in this case
% You must have a proper ".bib" file
%  and remember to run:
% latex bibtex latex latex
% to resolve all references

\end{document}  