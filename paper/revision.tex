\documentclass[11pt, oneside]{article}   	% use "amsart" instead of "article" for AMSLaTeX format
\usepackage{geometry}                		% See geometry.pdf to learn the layout options. There are lots.
\geometry{letterpaper}                   		% ... or a4paper or a5paper or ... 
%\geometry{landscape}                		% Activate for for rotated page geometry
%\usepackage[parfill]{parskip}    		% Activate to begin paragraphs with an empty line rather than an indent
\usepackage{graphicx}				% Use pdf, png, jpg, or eps with pdflatex; use eps in DVI mode
								% TeX will automatically convert eps --> pdf in pdflatex		
\usepackage{amssymb}

\title{Working document for revision of ICDE '14 paper}
\author{Kostas, Martin, Marcos}
%\date{}							% Activate to display a given date or no date

\begin{document}
\maketitle

\section{About}
In this document we write down the arguments we will use to address the desiderata agreed upon by Kostas and Marcos during a Skype meeting October 16 2013 (recorded in chat). Each section below describes a task or states question that we will provide an answer for in the paper. The answers are recorded in this document in a subsection below each question

\section{Question: Why map conflict resolution to an NP-hard problem?}
Here we assume that using conflict sets is a good idea. The question whether that is true is a separate question, that we will also answer.
\begin{itemize}
\item Definition of conflict resolution: Given a set of conflict sets $C$, and a number $\lambda_c < \left\vert c \right\vert$ for each set $c \in C$, choose $\lambda_c$ elements from each set $c$ that will be ``deleted'' 
\item Global optimization version: choose $\lambda_c$ elements of minimum total weight from each set $c$ that will be ``deleted'' 
\item This optimization problem equals set multicover problem, plain and simple. I don't see that we are even ``mapping'' to this problem. It is the same problem?
\item Is there another optimization problem (not minimizing total weight of elements deleted) that involves conflict sets and generates useful solutions to generalization?
\item The SGA algorithm is an exact algorithm for another optimization problem (minimize for each set); but is that really the ``right'' problem to solve and why not?
\end{itemize}

\subsection{Our answer}

\begin{itemize}
\item Actually, for most instances the bulk of the time is spent finding conflicts, not solving the set multicover problem, because in the end we don't solve SMCP
\item We apply approximation algorithms and heuristics which are exact solutions to various other problems that somehow relate to SMCP
\end{itemize}

\section{Question: Why use conflict sets to model problem?}

\begin{itemize}
\item Is there a "conflict free" formulation that would be just as ``good'' and not lead to solving an NP-hard problem?
\item Conflict sets allow a generic way to express many natural user-defined constraints (proximity, visibility)
\item Using conflict sets is a natural way to think about generalization
\item Maybe there is another model that could express proximity and visibility constraints, and which is a natural way to think about generalization, and which leads to an optimization problem that has a polynomial time algorithm?
\end{itemize}

\subsection{Our answer}

\begin{enumerate}
\item We chose conflict sets because it is a natural way to think about generalization and because it allows users to formulate several natural spatial constraints (proximity, visibility etc)
\item We can not think of another model with these properties and which also leads to an optimization problem that has an efficient solution
\item Asking us to prove that such a model does not exist is a tall order...
\end{enumerate}

\section{Questions: Various questions answered in author feedback}

Use this section to capture some compact formulations that can be plugged into paper

\end{document}  