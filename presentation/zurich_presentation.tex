\documentclass{beamer}
\usepackage{listings}
\usetheme{Copenhagen}
\usecolortheme{beaver}
\setbeamertemplate{navigation symbols}{}
\setbeamertemplate{footline}{\parbox[t][12pt][c]{12pt}{~\scriptsize\insertframenumber}}
% \usepackage{beamerthemesplit} // Activate for custom appearance

%%%%%%%%%%%%
% MVS: Language definitions
%
\renewcommand{\ttdefault}{pcr}
\lstset{
  basicstyle=\small\ttfamily,
  breaklines=true
}
\lstdefinelanguage{cvl}{
  morekeywords={generalize,to, with, other, at, zoom, levels, weigh, by, subject, and, create, constraint, as, not, exists, resolve, if, delete, select, from, where, in, order, over, setup, teardown,force,min,level,for,allornothing,join,on,setup,group,having,index,temporary,table,drop,partition,merge,partitions},
  sensitive=false,
  morecomment=[l]{//},
  morecomment=[s]{/*}{*/},
  morestring=[b]",
}
\lstset{
  language=cvl
}



\title{Declarative Cartography}
\subtitle{In-Database Map Generalization of Spatial Datasets}
\author{\underline{Pimin Konstantin Kefaloukos}, Marcos Vaz Salles, Martin Zachariasen\\ \small{Computer Science Department, University of Copenhagen} (DIKU)}
\date{\today}

\begin{document}

\frame{\titlepage}

% MOTIVATION
\frame
{
  \frametitle{Overview}
  \begin{center}
  \fbox{A narrow view of the world of digital web maps :-)}
  \end{center}
  
  \begin{itemize}
  \item \textbf{OK}: Background maps are stable and of good quality (commercial and national map providers)
  \item \textbf{Work needed}: Foreground data is dynamic and constantly emerging and changing (social media, sensors, ...)
  \item \textbf{Challenge}: Design \emph{easy-to-use}, \emph{cheap}, \emph{scalable} and \emph{effective} method for \underline{generalizing}, \underline{serving} and \underline{visualizing} foreground data
  \item \textbf{Use cases}: Data journalism, citizen information systems, crisis response, social media, ...
  \end{itemize}

}

\frame
{
  \frametitle{Map Generalization 101}
  Going from a big pile of data:
  \begin{center}
  \fbox{\includegraphics[scale=0.14]{figs/toomanyobjects.png}}
  \end{center}
  To a friendly map:
  \begin{center}
  \includegraphics[scale=0.19]{figs/generalized-tourism.pdf}
  \end{center}
}

\frame
{
  \frametitle{Our approach}
  \begin{itemize}
  \item \textbf{Language}: Design and implement an easy to use \emph{programming language} for generalizing spatial data 
  \item \textbf{Optimization}: Fit generalization problem to well-known \emph{optimization problem}, i.e. variant of Set Cover Problem~\cite{vazirani}
  \item \textbf{Algorithms}: Reuse \emph{existing algorithms} for this problem
  \item \textbf{Databases}: Reuse \emph{database technology} and \emph{move code to data} by implementing algorithms in a spatial database
  \end{itemize}
  \begin{center}
  \includegraphics[scale=0.3]{figs/indatabase-execution.pdf}
  \end{center}
}

\frame
{
  \frametitle{Main idea}
  \begin{itemize}
  \item \textbf{Language}: Design and implement an easy to use \emph{programming language} for generalizing spatial data 
  \item \textbf{Optimization}: Fit generalization problem to well-known \emph{optimization problem}, i.e. variant of Set Cover Problem~\cite{vazirani}
  \item \textbf{Algorithms}: Reuse \emph{existing algorithms} for this problem
  \item \textbf{Databases}: Reuse \emph{database technology} and \emph{move code to data} by implementing algorithms in a spatial database
  \end{itemize}
  \begin{center}
  \fbox{\includegraphics[scale=0.18]{figs/toomanyobjects.png}}
  \end{center}
}


%\frame
%{
%  \frametitle{Relation to Reverse Data Management}

%  We implemented \emph{how-to} queries for spatial data:
%  \begin{itemize}
%  \item \textbf{How-to queries}~\cite{reversedatamanagement}: Given an \emph{input database}; compute an \emph{output database}, subject to set of \emph{constraints} and minimizing/maximizing an \emph{objective function}
%  \item \textbf{Spatial example}: Given an \emph{input database of spatial objects}; compute an output database that generalizes the objects to $\mathcal{Z}$ zoom-levels, subject to \emph{spatial constraints} and while \emph{minimizing loss} of objects in map (with prioritization)
%  \end{itemize}
%}

\frame
{
  \frametitle{Please note!}
  Whenever you see the word \emph{``delete''} in this presentation, please append the words \emph{``or aggregate (future work)''} in your mind. \\
  making grammatical adjustments where appropriate.
  
  \begin{center}
  \includegraphics[scale=0.50]{figs/cvl-delete-aggregate.pdf}
  \end{center}

}


\frame
{
  \frametitle{Problem definition}
  \begin{itemize}
  \item Given set of spatial objects with \emph{unique ID}, \emph{weight} (models ``importance'' of object) and \emph{geometry} (point, polygon)
  \item Set of zoom-levels $\lbrace 1, 2, \dots, \mathcal{Z} \rbrace$
  \item Compute minimum zoom-level for all objects:
  \begin{itemize}
  \item Objective: Minimize weight of missing objects
  \item Subject to: Spatial constraints (proximity, density, ...)
  \end{itemize}
  \end{itemize}
  \begin{center}
  \fbox{\includegraphics[scale=0.4]{figs/cvl-problem.pdf}}
  \end{center}
}

\frame
{
  \frametitle{Constraints}

  \begin{itemize}
  \item \textbf{Zoom-consistency} Objects never disappear as we zoom in~\cite{fusiontables}
  \item \textbf{User-defined constraints}:
  \begin{itemize}
  \item Arbitrary property that must hold for \emph{sets of objects}
  \item If property does not hold for set of objects $\implies$ \emph{conflict}
  \item Conflicts resolved by \emph{deleting} objects
  \item Property example: \emph{proximity}
  \end{itemize}
  \end{itemize}
  
  \begin{center}
  	\fbox{\includegraphics[scale=0.20]{figs/cvl-proximity-conflicts-2.pdf}} \fbox{\includegraphics[scale=.38]{figs/gladiators.jpg}} \fbox{\includegraphics[scale=.288]{figs/the-godfather-1.jpg}}
  \end{center}
  
    \begin{itemize}
  \item Like a gladiator fight, but outcome is decided by ``management'' :-)
  \end{itemize}

}

\frame
{
  \frametitle{Example: Proximity}
  \textbf{Proximity constraint}:
  \begin{itemize}
  \item Any two objects must be separated by minimum distance
  \item Resolve conflicts by deleting one object
  \end{itemize}
  \begin{center}
  \fbox{\includegraphics[scale=0.4]{figs/cvl-proximity.pdf}} \\
  \small{Red objects violate proximity constraint}
  \end{center}
}

\frame
{
  \frametitle{Example: Visibility}
  \textbf{Visibility constraint}:  
  \begin{itemize}
  \item Within unit area, at most $K$ objects may be visible
  \item Resolve conflict (involving $K'$ objects) by deleting $K' - K$ objects
  \end{itemize}
  \begin{center}
  \includegraphics[scale=0.4]{figs/cvl-visibility.pdf} \\
  \small{Red objects violate visibility constraint ($K=8$)}
  \end{center}
}

\begin{frame}[fragile]
\frametitle{Declarative Language for Generalization (CVL)}

\begin{itemize}
\item Language has two statements: \emph{Generalize} and \emph{Create Constraint}
\item A \emph{Generalize} statement can reference constraints defined in a \emph{Create Constraint} statement
\item Language is visually similar to SQL and uses snippets of embedded SQL
\item Designed to be compiled into SQL and executed in a database (where the data is assumed to be stored)
\end{itemize}
\end{frame}


\begin{frame}[fragile]
\frametitle{CVL: Generalize statement}
\begin{lstlisting}
GENERALIZE 
   restaurants TO restaurants_generalized
AT 20 ZOOM LEVELS
WEIGH BY
  star_rating
SUBJECT TO 
   proximity 10 AND
   visibility 64
\end{lstlisting}

\begin{itemize}
\item \texttt{restaurants} is an existing database table
\item \texttt{restaurants\_generalized} will be computed  
\item \texttt{star\_rating} used as object weight (any valid SQL expression)
\item \texttt{proximity} and \texttt{visibility} references constraints
\end{itemize}

\end{frame}

\begin{frame}[fragile]
\frametitle{CVL: Create Constraint}
\begin{lstlisting}
CREATE CONSTRAINT {Name, e.g. Proximity}
AS NOT EXISTS
  {SQL select statement}
  
RESOLVE cid IF DELETE (
  {integer expression}
)
\end{lstlisting}

\begin{itemize}
\item Select statement should return two result columns: \texttt{cid} (conflict ID) and \texttt{rid} (object ID)
\item Proximity constraint: the SQL select statement is a two-way spatial distance join that finds all objects that are too close
\item Proximity constraint: the integer expression is the constant $1$
\end{itemize}

\end{frame}

\begin{frame}[fragile]
\frametitle{Defining the proximity constraint (experts only)}
\begin{lstlisting}
CREATE CONSTRAINT Proximity
AS NOT EXISTS (
  SELECT l.{rid} || r.{rid} AS cid, Unnest(array[l.{rid}, r.{rid}]) AS rid
  FROM
    {level_view} l JOIN {level_view} r
  ON
    l.{rid} < r.{rid}
  AND
    l.{geom} && ST_Expand(r.{geom}, CVL_Resolution({z}, 256) * {parameter_1})
  AND
    ST_Distance(l.{geom}, r.{geom}) < CVL_Resolution({z}, 256) * {parameter_1}
)

RESOLVE cid IF DELETE (
  1
)
\end{lstlisting}
\end{frame}

\begin{frame}[fragile]
\frametitle{Proximity constraint: the important part}
\begin{itemize}
\item The core of the proximity constraint is a two-way spatial join:
\end{itemize}
\begin{lstlisting}
SELECT
   ...
FROM 
   ... left JOIN ... right
ON 
   ST_Distance(left.the_geometry, right.the_geometry) < {meters}
\end{lstlisting}
\end{frame}





\frame
{
  \frametitle{Future work}
  \begin{itemize}
  \item Relax zoom-consistency: \emph{min} and \emph{max} zoom-level for records)
  \item Extend work with \emph{aggregation} in language (semantics) and algorithm (implementation)
  \item Implement compilation of CVL to a distributed database dialect
  \end{itemize}
}

\frame
{
  \frametitle{Future work: Aggregation}
  \begin{center}
  \fbox{This is where we need your help :-)}
  \end{center}

  \begin{itemize}
  \item \textbf{User-level semantics}:
  \begin{itemize}
  \item How should user \emph{understand} aggregation, e.g. aggregation of non-geometric attributes?
  \item How should user \emph{parameterize} aggregation?
  \end{itemize}
  \item \textbf{Implementation}:
  \begin{itemize}
  \item Can we \emph{stretch} the existing model to include aggregation?
  \item Else, how does aggregation \emph{change} the problem/model?
  \item Do we need new \emph{algorithms}?
  \end{itemize}
  \end{itemize}

  \begin{center}
  \fbox{\textbf{Eiffel Tower + Louvre = ?}}
  \end{center}

}



% RELATED WORK
\frame
{
  \frametitle{Related work}

  \begin{itemize}
  \item \emph{Efficient Spatial Sampling of Large Geographical Tables}. Das Sarma, A., Lee, H., Gonzalez, H., Madhavan, J., \& Halevy, A. (2012).
  \item \emph{Reverse data management}, Meliou, A., Gatterbauer, W., \& Suciu, D. (2011).
  \item \emph{Generalization of land cover maps by mixed integer programming}. Haunert, J.-H., \& Wolff, A. (2006). 
  \item \emph{Constant information density in zoomable interfaces}. Woodruff, A., Landay, J., Stonebraker, M. (1998).
  \item And tons more of course...
  \end{itemize}
}

% Past and future work
\frame
{
  \frametitle{Past and future work}

  \begin{itemize}
  \item Past work: \emph{TileHeat}, predicting where people will look on a map tomorrow
  \item Latest work: \emph{Declarative Cartography}, the work described in these slides
  \item Future work: \emph{Real-time Declarative Cartography}, joint work with people at University of Zurich (Department of Geography)
  \item Future work: Succinct data representation of high-fidelity spatial data that is visualized on a digital map on a screen (think: pixel precision is not all that good)
  \end{itemize}
}




\end{document}
