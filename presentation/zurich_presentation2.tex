\documentclass{beamer}
\usepackage{listings}
\usetheme{Copenhagen}
\usecolortheme{beaver}
\setbeamertemplate{navigation symbols}{}
\setbeamertemplate{footline}{\parbox[t][12pt][c]{12pt}{~\scriptsize\insertframenumber}}
% \usepackage{beamerthemesplit} // Activate for custom appearance

%%%%%%%%%%%%
% MVS: Language definitions
%
\renewcommand{\ttdefault}{pcr}
\lstset{
  basicstyle=\small\ttfamily,
  breaklines=true
}
\lstdefinelanguage{cvl}{
  morekeywords={generalize,to, with, other, at, zoom, levels, weigh, by, subject, and, create, constraint, as, not, exists, resolve, if, delete, select, from, where, in, order, over, setup, teardown,force,min,level,for,allornothing,join,on,setup,group,having,index,temporary,table,drop,partition,merge,partitions},
  sensitive=false,
  morecomment=[l]{//},
  morecomment=[s]{/*}{*/},
  morestring=[b]",
}
\lstset{
  language=cvl
}


\title{Declarative Cartography}
\subtitle{In-Database Map Generalization of Spatial Datasets}
\author{\underline{Pimin Konstantin Kefaloukos}, Marcos Vaz Salles, \\Martin Zachariasen\\ \small{\emph{Computer Science Department (DIKU)}, \textbf{University of Copenhagen}}}

\date{\today}

\begin{document}

\frame{\titlepage}

\frame
{
  \frametitle{Motivation}
  \begin{itemize}
  \item \textbf{Imagine}: you're a \emph{journalist} and want to tell a story about restaurants in Z\"{u}rich \emph{using a map}.
  \item \textbf{Database}: You have a \emph{database} of restaurants (unique ID, location, star rating, name etc.)
  \item Simply showing all the records creates a mess (see picture)
  \item Generalization of thematic data is an important problem, with increasing use cases in \emph{social networks}, \emph{data journalism} etc
  \end{itemize}
  \includegraphics[scale=0.5]{figs/spatial-database-with-points.pdf} \includegraphics[scale=0.18]{figs/zurich-unfiltered.pdf}
}

% Setting stage
\frame[t]
{
  \frametitle{What a good (thematic) map might look like}
\begin{columns}[t]
	\begin{column}[l]{6cm}
		\begin{itemize}[<+->]
			\item Records should appear \emph{gradually} when zooming in, i.e. \emph{constant information~density}
			\item ``Important'' records should get priority
			\item Visible records should \emph{remain visible} when zooming in
			\item There should be some way of controlling \emph{layout}
		\end{itemize}
	\end{column}
	\begin{column}[l]{6cm}
    \begin{figure}
    	\includegraphics<1>[scale=0.18]{figs/zoom12.pdf}
        \includegraphics<2>[scale=0.18]{figs/zoom13.pdf}
        \includegraphics<3>[scale=0.18]{figs/zoom14.pdf}
        \includegraphics<4>[scale=0.18]{figs/zoom15.pdf}
    \end{figure}                            
\end{column}
\end{columns}
}

% Setting stage
\frame[t]
{
  \frametitle{Spatial databases}
	\begin{itemize}[<+->]
		\item Spatial data is often stored in a \emph{spatial database}
		\item They have powerful capabilities
		\item \emph{joins}, \emph{sorting}, \emph{spatial indexing} and \emph{spatial functions}
	\end{itemize}
	\begin{center}
	\begin{figure}
    		\includegraphics<1>[scale=0.3]{figs/cvl-powerful-database.pdf}
    		\includegraphics<2>[scale=0.3]{figs/cvl-powerful-database.pdf}
		\includegraphics<3>[scale=0.3]{figs/cvl-powerful-database-2.pdf}
	\end{figure}       
	\end{center}                     
}

\frame[t]
{
  \frametitle{Situation}
	\begin{itemize}[<+->]
		\item Normally data is pulled out of the \emph{database} for processing tasks like generalization
		\item Generalization algorithms are usually implemented in external software
		\item After the processing is done the data is put back into the database
		\item \emph{Several drawbacks}: re-inventing the wheel, memory and bandwidth limits, development and deployment costs, 
	\end{itemize}
	\begin{center}
	\begin{figure}
    		\includegraphics<1>[scale=0.3]{figs/cvl-external-software-out.pdf}
    		\includegraphics<2>[scale=0.3]{figs/cvl-external-software-proc.pdf}
		\includegraphics<3>[scale=0.3]{figs/cvl-external-software-in.pdf}
		\includegraphics<4>[scale=0.3]{figs/cvl-external-software-rest.pdf}
	\end{figure}       
	\end{center}                     
}

\frame[t]
{
  \frametitle{Why not use the database itself?}
	\begin{itemize}[<+->]
		\item Instead, we state generalization in terms of a high-level language
		\item The language is translated to a low-level database program
		\item We move code-to-data instead of moving data-to-code

	\end{itemize}
	\begin{center}
	\begin{figure}
    		\includegraphics<1>[scale=0.3]{figs/cvl-state-intension.pdf}
    		\includegraphics<2>[scale=0.3]{figs/cvl-compile-statement.pdf}
		\includegraphics<3>[scale=0.3]{figs/cvl-code-to-data.pdf}
	\end{figure}       
	\end{center}                     
}


\frame[t]
{
  \frametitle{In-database processing}
	\begin{itemize}
		\item The low-level program can of course exploit capabilities of spatial databases, such as indexing and spatial functions
	\end{itemize}
	\begin{center}
	\begin{figure}
    		\includegraphics[scale=0.4]{figs/cvl-code-to-data-exploit.pdf}
	\end{figure}       
	\end{center}                     
}

\frame[t]
{
  \frametitle{The language}
	\begin{itemize}[<+->]
		\item High-level language has a \texttt{GENERALIZE} statement and a \texttt{CREATE CONSTRAINT} statement
		\item The \texttt{GENERALIZE} statement produces a generalized database table from a table of base data
		\item The \texttt{CREATE CONSTRAINT} statement is used to define new spatial constraints, e.g. proximity
		\item The table produced by \texttt{GENERALIZE} has a \texttt{min\_zoom} column added to all records
	\end{itemize}
	\begin{center}
  		\fbox{\includegraphics[scale=0.30]{figs/cvl-problem.pdf}}
  	\end{center}
}

% Introduce syntax
\begin{frame}[fragile,t]
  \frametitle{Language example}
  \begin{description}[<+->]
  \item
  \begin{lstlisting}
GENERALIZE restaurants 
\end{lstlisting}
  \item
\begin{lstlisting}
TO restaurants2
\end{lstlisting}
  \item
\begin{lstlisting}
AT 20 ZOOM LEVELS
\end{lstlisting}
  \item
\begin{lstlisting}
WEIGH BY star_rating
\end{lstlisting}
  \item
\begin{lstlisting}    
SUBJECT TO proximity 10 AND density 64
\end{lstlisting}    
  \end{description}
\end{frame}

% Introduce syntax
\begin{frame}[fragile,t]
  \frametitle{Language example}
  \begin{itemize}
  \item Result of generalization viewed at zoom-level 16
  \end{itemize}
  \begin{center}
  	\includegraphics[scale=0.30]{figs/zoom16.pdf}
  \end{center}

\end{frame}


\frame[t]
{
  \frametitle{Computing solutions}
	\begin{itemize}[<+->]
	\item Combination of record weights and spatial constraints (proximity, density) present a natural optimization problem
	\item Represent generalization problem as instances of set multicover problem (SMP)
	\item Reuse existing algorithms for SMP (database implementations)
	\item Solution to SMP $\rightarrow$ records that should be filtered out at given zoom level
	\end{itemize}
	\begin{center}
  		\includegraphics[scale=0.70]{figs/cvl-spatial-to-nonspatial.pdf}
  	\end{center}
}

\frame[t]
{
  \frametitle{Computing solutions}
	\begin{itemize}[<+->]
	\item Combination of record weights and spatial constraints (proximity, density) present a natural optimization problem
	\item Represent generalization problem as instances of set multicover problem (SMP)
	\item Reuse existing algorithms for SMP (database implementations)
	\item Solution to SMP $\rightarrow$ records that should be filtered out at given zoom level
	\end{itemize}
	\begin{center}
  		\includegraphics[scale=0.70]{figs/cvl-spatial-to-nonspatial.pdf}
  	\end{center}
}

\end{document}